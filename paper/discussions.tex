\section{Discussion}
\label{sec:discussion}

The result of this experiment shows possible gains by increasing the initial
congestion window for short bursty connection. However the topology selected for
testing is rather simple compared to the internet. It does not take congested
overloaded networks into account. High \emph{initcwnd} might hurt network
stability in this case, because congestion is perceived later. In our network
the bandwidth on all links was symmetric. Often client networks are
asymmetricrically connected to the internet. Clients only need a fraction of
uplink speed to acknowledge packages in case of downlink oriented traffic.
However in some cases if upload capacity is shared with other flows, queues are
filled up and causes long delays. This phenomenon is called buffer
bloat~\cite{rfc970}. Longer congestion windows will have a positive effect in
case the downlink is not congested. In the simulation only cubic was considered
as congestion algorithm. Cubic is known to be more conservative about increasing
the window as we see in figure~\ref{fig:cwnd_tcp_algos}, other algorithms such
as TCP Reno converge faster to higher sizes.
