\begin{abstract} The initial congestion window is an important parameter of the
  TCP slow start algorithm. It defines the upper limit of outstanding bytes a
  sender assumes at the beginning of new connection.

The current standard RFC3390~\cite{rfc3390} from 2002 specifies a window of four
segments (over 4KB). Large-scale experiments~\cite{36640} by Google from 2010
have shown that an increase of the initial congestion window improves the
overall performance, latency in particular, of web servers without a
negative impact on the network stability. The reasoning made by this research is
that most web requests only transmit a small amount of data and take a short
time. Therefor these connections do not live long enough to build up a congestion
window big enough to make efficient use of the available bandwith. Driven by
these results, the experimental standard RFC6928~\cite{rfc6928} was made to
propose an increase of the TCP's Initial Window by 10 segments.

In this paper the effect of different TCP's Initial Window on user latency is
evaluated in a controlled environment simulating various network condition with
the use of Mininet~\cite{mininet}.
\vspace{-4mm}
\end{abstract}
