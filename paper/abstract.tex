\begin{abstract}
The initial congestion window is an important parameter of the TCP slow start
algorithm. It defines the upper limit of outstanding bytes a sender assumes at
the beginning of new connection.

The current standard RFC3390~\cite{rfc3390} from 2002 specifies a window of four
segments (over 4KB). Large-scale experiments~\cite{36640} by Google from 2010
have shown that an increase of the initial congestion window improves the
overall performance of web servers without a negative impact on the network
stability. In their research they measured transmission time for web requests to
their services. These only transmit a small amount of data and take a short
time. Therefor the tcp connections do not live long enough to build up a
congestion window big enough to make efficient use of the available bandwidth.
Driven by these results, the experimental standard RFC6928~\cite{rfc6928} was
made to propose an increase of the TCP's Initial Window by 10 segments.

In this paper the effect of different TCP's Initial Window on user latency is
evaluated in a controlled environment simulating various network condition with
the use of Mininet~\cite{mininet}.

First we present the TCP congestion control mechanisms and how certain network
conditions influence throughput and transmission time. Then we show our
measurement methodology. After that we will evaluate our results and discuss the
outcomes.
\vspace{-4mm}
\end{abstract}
