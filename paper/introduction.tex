\section{Introduction}
\label{sec:introduction}

The Transmission Control Protcol, short TCP, is most popular communication
protcol on the Internet. To make it scaleable and adaptive to the current
condition of network without overloading it, congestion control mechanism are
used. Part of these algorithm is TCP slow start as decribed in
RFC5681~\cite{rfc5681}.

The TCP slow algoritm limits the number of bytes it sends on a new tcp
connection before waiting for an acknowledgement. On every acknowleged packet
the sender increases the window size by one segment until the Slow-start
threshold (\emph{ssthres}) is reached  or the receiver advertises a lower
receiver window (\emph{rwnd}) in its tcp header. The exponential grow of the
window every roundtrip will be also stopped if TCP detects packet loss. This
could be indicated by either a duplicate acknowlegdment of the receiver or a
timeout. Depending on the congestion algorithm in use the sender will then
behave differently.


. The initial
congestion window therefor defines the upper limit of bytes a sender assumes at
the beginning.
