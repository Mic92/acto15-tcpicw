\section{Evaluation}
\label{sec:evaluation}

We depicted time relative to a window of 3 for selected values in this paper for
1MBps, 5MBps and 100MBps in figures \ref{fig:bandwith-1mb},
\ref{fig:bandwith-5mb} and \ref{fig:bandwith-100mb}. Graphs and raw data of the
whole measurement can be also looked up on this page:
\url{https://mic92.github.io/acto15-tcpicw/}

As expected changing the initial window size for rather small and big request
sizes did not result in an improvement of request time. For small sizes
($\leq{}4\text{kiB}$) the initial window of 3 is already sufficient to transmit
all values within 1 round trip time. For bigger requests
($\gtrsim{}2048\text{kiB}$) TCP Slow Start algorithm had enough time to scale up
to the bandwidth limit.

If we look at transmission sites more common for web request (16kiB-256kiB), we
the see performance improves between 10\% and 20\% for a initial TCP window of
10 segments. The saving is higher, if the link latency gets higher or the link
provides more bandwidth. Therefore connections with a high
Bandwidth-Delay-Product profit most from a bigger initial congestion window.
This behaviour is a consequence of equation~\ref{transfer_time}. Improving the
window size beyond 10 segments still improves the user latency so it saves up
30\%-40\% of time for window sizes like 32 and 40. However the scaling is not
linear, which is also consistent with the model~\ref{transfer_time}.
